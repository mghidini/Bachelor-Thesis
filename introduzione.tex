\chapter*{Introduzione}
\label{ch:introduzione}

Nel contesto dell'evoluzione tecnologica e dell'importanza crescente dei dati nel settore sanitario, le applicazioni web in ambito medico rappresentano una risorsa importante per migliorare la qualità delle cure mediche e ottimizzare i processi decisionali clinici.
Esse, infatti, semplificano la comunicazione medico-paziente, favoriscono la comprensione dei dati e facilitano il calcolo e la predizione, anche dove entrano in gioco dati in grande quantità e di elevata complessità.

Il presente lavoro di tesi si propone di esplorare tale contensto, attraverso lo sviluppo e l’implementazione di un Sistema di Supporto Decisionale (Decision Support System o DSS), partendo da un applicativo già esistente, Epimetheus.
\newline
Il contesto è, dunque, quello medico e, più precisamente, quello chirugico-ortopedico.
\newline
L'applicazione Epimetheus fornisce predizioni sull'esito di operazioni chirurgiche ortopediche, in particolare riguardo alla ripresa del paziente, assistendo così il medico nel processo decisionale clinico.
In particolare, il supporto si propone di evidenziare un aspetto molto delicato di quello che è il percorso di decision-making, ovvero quello della rappresentazione dell’incertezza.

\section*{Obbiettivo della tesi}
\label{sec:obbiettivo}
Con questa relazione, si vuole descrivere quanto svolto durante l’esperienza di stage presso l’Università degli Studi Milano Bicocca, in collaborazione con il laboratorio MUDI, del Dipartimento di Informatica, Sistemistica e Comunicazione (DISCo).
\newline
L'obiettivo dello stage riguarda lo sviluppo e deployment dell’applicazione web denominata Epimetheus, che implementa un Decision Support System in ambito di decision-making medico, con particolare attenzione agli aspetti di sviluppo back-end e di flusso logico. L'implementazione è in linguaggio Python ed è volta all'integrazione di componenti di Machine Learning preesistenti nell'applicativo sopra descritto.

La struttura della tesi è organizzata come segue: nel primo capitolo verrà approfondito lo stato dell'arte riguardante le applicazioni dell'intelligenza artificiale e i sistemi di supporto decisionali, soffermandosi anche sul concetto di incertezza; nel secondo capitolo verrà descritta la metodologia adottata nel progetto, dall’analisi dell’applicazione pre-esistente, alla progettazione e implementazione della nuova architettura back-end, alla messa in produzione dell’applicazione; infine, verranno esposte le conclusioni e i possibili sviluppi futuri per il progetto.