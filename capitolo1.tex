\chapter{Contesto teorico}
\label{ch:capitolo1}

% --- Inizio del Capitolo 1 ---

\section{AI e Machine Learning in ambito medico}
\label{sec:AIambitoMedico}

L’Intelligenza Artificiale e il Machine Learning stanno gradualmente rafforzando il loro impatto nella vita di tutti i giorni, e prendono sempre più posizione nel campo della sanità, influenzando il processo di diagnostica, prevenzione e cura della malattia.
\newline
L'impiego nel settore potrebbe rivoluzionare il sistema sanitario, rendendolo sempre più accurato, sicuro e veloce.

Nell’articolo \textit{“Clinical applications of artifcial intelligence and machine learning in cancer diagnosis: looking into the future”}\cite{cancerDiagnosis}, Muhammad Javed Iqbal \textit{et al.} analizzano gli sviluppi dell’AI nell’ambito oncologico.
\newline
È noto come le tecnologie di imaging medico (MIT) assistite da Intelligenza Artificiale, abbiano un ruolo fondamentale, in quanto sono in grado di interpretare e classificare immagini mediche, organizzare i dati conseguentemente, archiviare informazioni ed eseguire data mining. Grazie a questo ampio spettro di azione, l’AI, che è già un supporto fondamentale per il settore di radiologia, può essere declinata anche in contesti oncologici.
\newline
Oltre alle tecnologie MIT, un altro impiego è quello della progettazione di algoritmi per la rilevazione precoce dei tumori, attraverso, ad esempio, l'identificazione di biomarcatori (indicatori di processi fisiologici, patologici o di risposte biologiche all'intervento terapeutico). Ciò è utile anche per sviluppare farmaci di precisione, che in oncologia sono progettati per mirare specificamente alle cellule tumorali. Risulterebbe quindi possibile suggerire terapie efficaci, considerando fattori genetici personalizzati. 
\newline
Pertanto, l'Intelligenza Artificiale può essere classificata tra le terapie futuristiche più avanzate per la diagnosi, la prognosi e il trattamento preciso dei tumori.

Secondo uno studio condotto da A. B. Lisacek-Kiosoglous e\textit{t al.}, \textit{“Artificial intelligence in orthopaedic surgery”}\cite{orthopaedic}, l’AI potrebbe giocare un ruolo fondamentale anche in ambito ortopedico, nel prossimo futuro.
\newline
Le applicazioni esistenti in chirurgia ortopedica hanno già evidenziato successi nell’individuare impianti in artroplastica e nel segnalare caratteristiche o problemi relativi ad essi, come il cattivo posizionamento o allentamento delle protesi.
\newline
Risultano, poi, di particolare importanza gli algoritmi predittivi di AI e ML, per ottenere, ad esempio, stime sulla lunghezza della convalescenza o permanenza in ospedale e sui costi. Inoltre, possono giocare un ruolo fondamentale nell'ambito di riabilitazione e trattamenti postoperatori, facendo previsioni sui risultati delle operazioni e lo stato dei pazienti.

Un’altra applicazione di AI e Machine Learning in ambito medico, riguarda i Sistemi di Supporto Decisionale, che forniscono un valido aiuto ai medici nel processo di decision-making, e che analizzeremo più approfonditamente nel prossimo paragrafo.

\section{Clinical Decision Support System}
\label{sec:CDSS}

Un Sistema di Supporto Decisionale (\textit{Decision Support System}) è un sistema informatizzato, che ha l’obbiettivo di assistere l’utente nel processo di \textit{decision making}, analizzando e combinando dati e informazioni specifiche in un determinato ambito e fornendo un aiuto per comprendere problemi, valutare alternative e prendere decisioni informate.

Nel caso dei \textit{Clinical Decision Support System} (CDSS), il contesto è quello medico; questo tipo di software rappresenta quindi uno strumento utile al personale sanitario. 
\newline
Le caratteristiche di un singolo paziente sono confrontate con una base di conoscenza clinica computerizzata e la situazione specifica dei pazienti o le raccomandazioni sono poi presentate al medico per una decisione. 
\newline
I CDSS oggi sono principalmente usati dal medico per combinare la propria conoscenza con i suggerimenti proposti dal sistema. Tuttavia, sempre più sistemi vengono sviluppati con la capacità di evidenziare dati e osservazioni altrimenti non ottenibili o non interpretabili dagli umani.
\newline
Attualmente, i CDSS fanno spesso uso di applicazioni web o integrazioni con cartelle cliniche elettroniche e CPOE (Immissioni computerizzate di ordini medici). Possono essere amministrati da desktop, tablet e smartphone, ma anche dispositivi di monitoraggio biometrico e tecnologia \textit{wearable}.
\newline
Il loro spettro di utilizzo è vasto e include, ad esempio, diagnostica, sistemi di allerta, prescrizioni e controllo di medicinali.

In un analisi svolta da Reed T. Sutton \textit{et al.}  sui CDSS, dal titolo \textit{"An overview of clinical decision support systems: benefits, risks, and strategies for success"}\cite{cdss}, vengono identificati tre componenti principali per questi sistemi:

\begin{enumerate}
    \item La \textbf{base di conoscenza}, costituita dalle regole implementate nel sistema o gli algoritmi utilizzati per modellare le decisioni e dai dati disponibili. 
    \item Il \textbf{motore di inferenza}, che combina le regole (programmate o modellate da AI) e le applica ai dati specifici del paziente, generando un output da presentare all'utente.
    \item Un \textbf{meccanismo di comunicazione}, per mezzo del quale il risultato finale viene mostrato all'utente; può essere un sito web, un’applicazione o un’interfaccia, tramite cui l’utente interagisce con il sistema.
\end{enumerate}

Sempre sulla base sullo studio specifico condotto da Reed T. Sutton \textit{et al.}\cite{cdss}, si andranno ora ad esporre alcuni benefici e rischi, derivanti dall'adozione di Sistemi di Supporto Decisionale in ambito sanitario.
\newline
Il primo vantaggio evidenziato riguarda la sicurezza del paziente; l’impiego di questi supporti, infatti, determina una riduzione dell'incidenza degli eventi avversi e degli errori, per esempio, nella prescrizione di farmaci.
\newline
Dal punto di vista delle organizzazioni sanitarie, si rileva un contenimento dei costi. Questo perché vengono notevolmente ridotte le duplicazioni di test e ordini, si possono sfruttare suggerimenti di opzioni di trattamento o farmaci più economici e vengono automatizzati passaggi tediosi, riducendo il carico di lavoro del personale medico.
\newline
Il beneficio maggiore deriva sicuramente dal fatto che i CDSS rappresentano un supporto diagnostico e decisionale per il medico e per il paziente stesso. Questi sistemi sono in grado di fornire suggerimenti diagnostici basati sui dati, interpretare risultati dei test di laboratorio e visualizzare e analizzare immagini mediche. Attraverso la comprensione e lo studio dei risultati proposti, medico e paziente hanno la possibilità di confrontarsi e compiere decisioni ponderate.
\newline
Per quanto riguarda gli svantaggi, si evidenzia innanzitutto un limite pratico, ovvero la necessità di competenze informatiche; spesso i CDSS possono richiedere conoscenze tecnologiche anche elevate per essere utilizzati, e ciò potrebbe precludere a una grande quantità di utenti la possibilità di beneficiare di questi sistemi.
\newline
Il secondo rischio riguarda invece la fiducia degli utenti e dei pazienti nei confronti di questo tipo di supporti. Da una parte ci può essere una mancanza di fiducia, che porta gli utilizzatori a non tenere in considerazione le linee guida  e i suggerimenti forniti dal CDSS. L’estremo opposto, altrettanto pericoloso, è la dipendenza o la fiducia eccessiva nell'accuratezza del software.

I CDSS sono sicuramente un mezzo per avere una visione oggettiva dello stato del paziente e possono fornire suggerimenti basati sull'evidenza. Tuttavia, la medicina è raramente un contesto semplice e sicuro. Per questo motivo è importante, nel \textit{decision-making} in ambito clinico, sviluppare sistemi che siano in grado di comunicare e rappresentare l’incertezza.

\section{Rappresentazione dell'incertezza in contesti medici}
\label{sec:incertezza}

La pratica della medicina non è mai un contesto semplice e assoluto. I dati che vengono utilizzati per prendere decisioni cliniche sono spesso ambigui, contraddittori o scarsi; perciò, un buon approccio richiede equilibrio tra l’esperienza del medico, le conoscenze disponibili e le preferenze del paziente.
\newline
L'incertezza è un elemento intrinseco alla pratica medica e influenza il processo delle decisioni di tipo clinico.

Nel suo elaborato di tesi, dal titolo \textit{"Comunicare l'incertezza in ambito medico. Data visualization di supporto alla prognosi di interventi chirurgici e al decision making"} \cite{tesiFrontino}, L. Frontino, sottolinea come, al giorno d’oggi, la maggior parte dei medici si sentano obbligati a fornire certezze assolute ai propri pazienti. Questo, oltre a causare problemi di comunicazione tra le due parti, contribuisce, insieme all'analfabetismo numerico, ad aggravare l’incapacità delle persone di comprendere e ragionare obbiettivamente sulle incertezze. Concentrandosi solo sulla ricerca di riscontri e risultati certi, infatti, si incorre facilmente in illusioni di certezza e ignoranza del rischio.
\newline
Bisogna ricordare, inoltre, che tra medico e paziente è presente un divario anche di tipo sociale, e che il paziente si trova spesso in una condizione di svantaggio rispetto al linguaggio utilizzato e alle conoscenze necessarie a comprendere informazioni in contesto medico.  
\newline
È quindi naturale, che si vada verso un sempre più attivo processo di collaborazione tra medico, paziente e strumenti tecnologici. Uno dei risultati di questa sinergia sono proprio i Sistemi di Supporto Decisionale, che sono in grado di costruire un ponte tra le  figure di medico e paziente, favorendo una comunicazione senza fraintendimenti.

Nell’articolo \textit{“Uncertainty in Decision-Making in Medicine: A Scoping Review and Thematic Analysis of Conceptual Models”},  di Helou M. A. \textit{et al.}, viene analizzato il ruolo dell'incertezza in ogni punto del processo di decision-making clinico: dalla prevenzione, alla diagnosi, fino alla scelta di cure e trattamenti.  Gli autori propongono degli step attraverso cui il medico si può confrontare con situazioni di incertezza:

\begin{enumerate}
    \item Riconoscimento dell'incertezza
    \item Classificazione dell'incertezza
    \item Considerazione di diverse prospettive e parti interessate
    \item Acquisizione di conoscenze
    \item Approccio alla decisione rispetto all'incertezza
\end{enumerate}

I CDSS, in particolare, possono essere un valido aiuto durante le fasi di questo processo; fornendo supporto decisionale basato sull'evidenza, essi aiutano i medici a valutare i rischi, considerare alternative e ponderare i benefici e i limiti di diverse opzioni di trattamento. 
\newline
Inoltre, incorporando modelli di ragionamento basati sulla probabilità, consentono di valutare l’occorrenza di diagnosi o esiti clinici, aiutando a prendere decisioni più informate e razionali, in contesti incerti e talvolta poco trasparenti.

Per quanto riguarda l’aspetto comunicativo, una buona comunicazione dell'incertezza passa anche attraverso la visualizzazione dei risultati forniti dai CDSS. Un buon supporto decisionale, infatti, deve avere come requisito la capacità di mostrare in modo chiaro e diretto i dati calcolati, in modo che siano facilmente leggibili sia dal medico che dal paziente.
\newline
A questo scopo è stato ideato il progetto Epimetheus, un \textit{Clinical Decision Support System} per l’ambiente chirurgico-ortopedico, di cui si discuterà più in dettaglio nel capitolo \ref{ch:capitolo2}.
