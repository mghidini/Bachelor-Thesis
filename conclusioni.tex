\chapter*{Conclusioni}
\label{ch:conclusioni}

Il risultato di questo progetto di tesi è stato lo sviluppo e l’implementazione del sistema Epimetheus, un \textit{Clinical Decision Support System} nel contesto della prognosi d'interventi chirurgici complessi, con l'obiettivo di visualizzare l'incertezza dei dati, attraverso predizioni, calcolate con modelli di Machine Learning.

Lo studio degli aspetti teorici ha evidenziato l’importanza e l’utilità di questo tipo di strumenti in ambito medico.

Dall'analisi condotta sull'applicativo, inoltre, è emersa la necessità di pensare ad una nuova architettura per il progetto, che prevedesse la separazione tra front-end e back-end; il server back-end, che implementa una API RESTful, comunica con il front-end tramite specifici endpoint. Questa ristrutturazione ha reso il sistema meno rigido e più scalabile, e quindi pronto ad eventuali ampliamenti.
\newline
Tra gli sviluppi futuri, potrebbero essere integrati nuovi modelli di ML, e quindi nuove procedure su cui poter ottenere risultati. In questo senso, sarebbe utile lavorare alla creazione di un vero e proprio database per mantenere le strutture dei form delle diverse procedure.
\newline
Un altro aspetto utile rispetto alla creazione di basi di dati legate al sistema, sarebbe la possibilità di mantenere eventuali dati riguardanti gli utenti e gli utilizzi del sistema, al fine di poter creare statistiche.

Infine, il deployment dell'applicazione su un server pubblico consente l’utilizzo di Epimetheus a utenti esterni, permettendo l’avvio di una nuova fase di testing. 
\newline
In questo contesto, una volta che la parte di front-end sarà pronta per essere messa in produzione, sarà possibile completare il deployment, integrando React con il server Apache in modo analogo a quanto è stato fatto con il back-end Flask.
\newline
Testare l’applicazione in un ambiente reale e ricevere \textit{feedback} dagli utenti evidenzierà i punti forti e le criticità dell'applicativo, favorendo uno sviluppo continuo ed iterativo del sistema.
\newline
L’utilizzo da parte del personale medico, poi, contribuirà a promuovere l'accettazione di questo tipo di supporti:  dimostrarne l’utilità e l’affidabilità nel contesto clinico, incoraggerebbe gli operatori sanitari ad utilizzarla e a integrarla nella loro pratica.
